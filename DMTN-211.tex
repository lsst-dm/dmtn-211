% Submit a 4-page-minimum manuscript, by the advertised due date, for publication in the Proceedings of SPIE in the SPIE Digital Library
\documentclass[]{spie}
\input{meta}

% Package imports go here.
\usepackage{amsmath,amsfonts,amssymb}
\usepackage{graphicx}
\usepackage[colorlinks=true, allcolors=blue, backref=page]{hyperref}
\usepackage{xspace}
\usepackage{longtable}
\usepackage{xcolor}
\usepackage{listings}
\usepackage{todonotes}

% Local commands go here.
\newcommand{\apj}{ApJ}
\newcommand{\apjs}{ApJS}
\newcommand{\procspie}{Proc.\ SPIE}
\newcommand{\pasj}{PASJ}
\newcommand{\degsq}{deg$^2$\xspace}
\newcommand{\faro}{\texttt{faro}\xspace}
\newcommand{\validatedrp}{\texttt{validate\_drp}\xspace}
\newcommand{\ro}{Rubin Observatory\xspace}
\newcommand{\lsst}{Legacy Survey of Space and Time (LSST)\xspace}
\newcommand{\squash}{\texttt{SQuaSH}\xspace}
\renewcommand{\baselinestretch}{1.0} % Change to 1.65 for double spacing

\newcommand{\docRef}{DMTN-211}
\newcommand{\docUpstreamLocation}{\url{https://github.com/lsst-dm/dmtn-211}}
\def\acronyms{\section*{ACRONYMS}\label{sec:acronyms}}

\definecolor{codegreen}{rgb}{0,0.6,0}
\definecolor{codegray}{rgb}{0.5,0.5,0.5}
\definecolor{codepurple}{rgb}{0.58,0,0.82}
\definecolor{backcolour}{rgb}{0.95,0.95,0.92}

\lstdefinestyle{farostyle}{
    backgroundcolor=\color{backcolour},   
    commentstyle=\color{codegreen},
    keywordstyle=\color{magenta},
    numberstyle=\tiny\color{codegray},
    stringstyle=\color{codepurple},
    basicstyle=\ttfamily\footnotesize,
    breakatwhitespace=false,         
    breaklines=true,                 
    captionpos=b,                    
    keepspaces=true,                 
    numbers=none,                    
    numbersep=5pt,                  
    showspaces=false,                
    showstringspaces=false,
    showtabs=false,                  
    tabsize=2
}
% \lstset{style=farostyle}

\newcommand\YAMLcolonstyle{\color{red}\mdseries}
\newcommand\YAMLkeystyle{\color{black}\bfseries}
\newcommand\YAMLvaluestyle{\color{blue}\mdseries}

\lstdefinelanguage{YAML}{
    keywords={true,false,null,y,n},
    keywordstyle=\color{darkgray}\bfseries,
    basicstyle=\YAMLkeystyle,                                 % assuming a key comes first
    sensitive=false,
    comment=[l]{\#},
    morecomment=[s]{/*}{*/},
    commentstyle=\color{purple}\ttfamily,
    stringstyle=\YAMLvaluestyle\ttfamily,
    moredelim=[l][\color{orange}]{\&},
    moredelim=[l][\color{magenta}]{*},
    moredelim=**[il][\YAMLcolonstyle{:}\YAMLvaluestyle]{:},   % switch to value style at :
    morestring=[b]',
    morestring=[b]",
    literate =    {---}{{\ProcessThreeDashes}}3
    {>}{{\textcolor{red}\textgreater}}1
    {|}{{\textcolor{red}\textbar}}1
    {\ -\ }{{\mdseries\ -\ }}3,
}

\title{Faro: A framework for measuring the scientific performance of petascale Rubin Observatory data products}
\input{authors}
\date{\today}

% Option to view page numbers
\pagestyle{plain} % change to \pagestyle{plain} for page numbers   
%\setcounter{page}{301} % Set start page numbering at e.g. 301
 
\begin{document}
\maketitle

\begin{abstract}

The Vera C.\ Rubin Observatory will advance many areas of astronomy over the next decade with its unique wide-fast-deep multi-color imaging survey, the Legacy Survey of Space and Time (LSST)\cite{2019ApJ...873..111I}.
The LSST will produce approximately 20TB of raw data per night, which will be automatically processed by the LSST Science Pipelines to generate science-ready data products -- processed images, catalogs and alerts. 
To ensure that these data products enable transformative science with LSST, stringent requirements have been placed on their quality and scientific fidelity, for example on image quality and depth, astrometric and photometric performance, and object recovery completeness. 
In this paper we introduce \faro, a framework for automatically and efficiently computing scientific performance metrics on the LSST data products for units of data of varying granularity, ranging from single-detector to full-survey summary statistics. 
By measuring and monitoring metrics, we are able to evaluate trends in algorithmic performance and conduct regression testing during development, compare the performance of one algorithm against another, and verify that the LSST data products will meet performance requirements by comparing to specifications. 
We present initial results using \faro to characterize the performance of the data products produced on simulated and precursor data sets, and 
 discuss plans to use  \faro to verify the performance of the LSST commissioning data products.
 
\end{abstract}


\keywords{Rubin Observatory, Legacy Survey of Space and Time, LSST, Data Management, Metric,  Verification}

%  todos
\newpage
\listoftodos
\newpage

% The main body of text

\section{References and Prior Art}

The LSST high-level science requirements are outlined in the SRD  \cite{LPM-17}. 

Vera C. Rubin Observatory Legacy Survey of Space and Time (LSST) Science Pipelines, 
\cite{2019ASPC..523..521B, 10.1093/pasj/psx080}

Introducing validate\_drp: Calculate SRD Key Performance Metrics for an output repository: 
\cite{DMTN-008}

The Rubin Observatory LSST Science Pipelines: 
 \cite{2019ASPC..523..521B} and \cite{10.1093/pasj/psx080}

LSST Science Pipeline Characterization Metric Report
\cite{DMTR-311}


\acknowledgments
This material is based upon work supported by the National Science Foundation under Cooperative Agreement 1258333 managed by the Association of Universities for Research in Astronomy (AURA), and the Department of Energy under Contract No. DE-AC02-76SF00515 with the SLAC National Accelerator Laboratory. 
Additional funding for Rubin Observatory comes from private donations, grants to universities, and in-kind support from LSSTC Institutional Members.


%The appendices
\appendix
%\section{Code Release}
The \faro code is distributed as part of the LSST Science Pipelines and released via GitHub (\url{https://github.com/lsst/faro}). 
\faro is also available as part of the LSST Science Pipelines Docker images available via Docker Hub (\url{https://hub.docker.com/r/lsstsqre/centos/}).
\faro includes a Python API and can be run from the command line via the LSST Science Pipelines. 
The documentation can be found online at \url{https://pipelines.lsst.io/faro}

\section{Faro Tutorial}
An online tutorial demonstrating use of \faro via the Python API can be found at (URL).
This tutorial takes as input, processed X data and computes a variety of metrics on the processed data products. 
Additionally, the tutorial exposes the use of several other components of the LSST Science Pipelines, notable the PipelineTask framework for <> (reference) and the Data Butler that abstracts access to the data  (ref Tim's paper)

\section{Faro Data Products}
What can I point to as a test dataset and data products? 
Detailing of data products? 



\acknowledgments

This material or work is supported in part by the National Science Foundation through Cooperative Agreement AST-1258333, Cooperative Support Agreement AST-1202910, and Cooperative Support Agreement AST-1836783 managed by the Association of Universities for Research in Astronomy (AURA), and support from Department of Energy under Contract No. DE-AC02-76SF00515 with the SLAC National Accelerator Laboratory managed by Stanford University. 
Additional Rubin Observatory funding comes from private donations, grants to universities, and in-kind support from LSSTC Institutional Members.
	
% Include all the relevant bib files.
% https://lsst-texmf.lsst.io/lsstdoc.html#bibliographies
\bibliographystyle{spiebib}
\bibliography{local,lsst,lsst-dm,refs_ads,refs,books}

\acronyms 
\addtocounter{table}{-1}
\begin{longtable}{p{0.145\textwidth}p{0.8\textwidth}}\hline
\textbf{Acronym} & \textbf{Description}  \\\hline

AURA & Association of Universities for Research in Astronomy \\\hline
DE & dark energy \\\hline
LSST & Legacy Survey of Space and Time (formerly Large Synoptic Survey Telescope) \\\hline
LSSTC & LSST Corporation \\\hline
NASA & National Aeronautics and Space Administration \\\hline
SLAC & SLAC National Accelerator Laboratory \\\hline
\end{longtable}


\end{document}