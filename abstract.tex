\begin{abstract}

The Vera C.\ Rubin Observatory will advance many areas of astronomy over the next decade with its unique wide-fast-deep multi-color imaging survey, the Legacy Survey of Space and Time (LSST) \cite{2019ApJ...873..111I}.
The LSST will produce approximately 20TB of raw data per night, which will be automatically processed by the LSST Science Pipelines to generate science-ready data products -- processed images, catalogs and alerts. 
To ensure that these data products enable transformative science with LSST, stringent requirements have been placed on their quality and scientific fidelity, for example on image quality and depth, astrometric and photometric performance, and object recovery completeness. 
In this paper we introduce {\ttfamily faro}, a framework for automatically and efficiently computing scientific performance metrics on the LSST data products for units of data of varying granularity, ranging from single-detector to full-survey summary statistics. 
By measuring and monitoring metrics, we are able to evaluate trends in algorithmic performance and conduct regression testing during development, compare the performance of one algorithm against another, and verify that the LSST data products will meet performance requirements by comparing to specifications. 
We present initial results using {\ttfamily faro} to characterize the performance of the data products produced on simulated and precursor data sets, and 
 discuss plans to use  {\ttfamily faro} to verify the performance of the LSST commissioning data products.
 
\end{abstract}

